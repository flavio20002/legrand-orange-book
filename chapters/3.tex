\chapter{Mathematics}

\section{Theorems}\index{Theorems}

\subsection{Several equations}\index{Theorems!Several Equations}

This is a theorem consisting of several equations.

\begin{theorem}[Name of the theorem] % Specify a name/title in square brackets, or leave them out for no title
	In $E=\mathbb{R}^n$ all norms are equivalent. It has the properties:
	\begin{align}
		& \big| ||\mathbf{x}|| - ||\mathbf{y}|| \big|\leq || \mathbf{x}- \mathbf{y}||\\
		&  ||\sum_{i=1}^n\mathbf{x}_i||\leq \sum_{i=1}^n||\mathbf{x}_i||\quad\text{where $n$ is a finite integer}
	\end{align}
\end{theorem}

\subsection{Single Line}\index{Theorems!Single Line}

This is a theorem consisting of just one line.

\begin{theorem} % Specify a name/title in square brackets, or leave them out for no title
	A set $\mathcal{D}(G)$ in dense in $L^2(G)$, $|\cdot|_0$. 
\end{theorem}

%------------------------------------------------

\section{Definitions}\index{Definitions}

A definition can be mathematical or it could define a concept.

\begin{definition}[Definition name] % Specify a name/title in square brackets, or leave them out for no title
	Given a vector space $E$, a norm on $E$ is an application, denoted $||\cdot||$, $E$ in $\mathbb{R}^+=[0,+\infty[$ such that:
	\begin{align}
		& ||\mathbf{x}||=0\ \Rightarrow\ \mathbf{x}=\mathbf{0}\\
		& ||\lambda \mathbf{x}||=|\lambda|\cdot ||\mathbf{x}||\\
		& ||\mathbf{x}+\mathbf{y}||\leq ||\mathbf{x}||+||\mathbf{y}||
	\end{align}
\end{definition}

%------------------------------------------------

\section{Notations}\index{Notations}

\begin{notation} % Specify a name/title in square brackets, or leave them out for no title
	Given an open subset $G$ of $\mathbb{R}^n$, the set of functions $\varphi$ are:
	\begin{enumerate}
		\item Bounded support $G$;
		\item Infinitely differentiable;
	\end{enumerate}
	a vector space is denoted by $\mathcal{D}(G)$. 
\end{notation}

%------------------------------------------------

\section{Remarks}\index{Remarks}

This is an example of a remark.

\begin{remark}
	The concepts presented here are now in conventional employment in mathematics. Vector spaces are taken over the field $\mathbb{K}=\mathbb{R}$, however, established properties are easily extended to $\mathbb{K}=\mathbb{C}$.
\end{remark}

%------------------------------------------------

\section{Corollaries}\index{Corollaries}

\begin{corollary}[Corollary name] % Specify a name/title in square brackets, or leave them out for no title
	The concepts presented here are now in conventional employment in mathematics. Vector spaces are taken over the field $\mathbb{K}=\mathbb{R}$, however, established properties are easily extended to $\mathbb{K}=\mathbb{C}$.
\end{corollary}

%------------------------------------------------

\section{Propositions}\index{Propositions}

\subsection{Several equations}\index{Propositions!Several Equations}

\begin{proposition}[Proposition name] % Specify a name/title in square brackets, or leave them out for no title
	It has the properties:
	\begin{align}
		& \big| ||\mathbf{x}|| - ||\mathbf{y}|| \big|\leq || \mathbf{x}- \mathbf{y}||\\
		&  ||\sum_{i=1}^n\mathbf{x}_i||\leq \sum_{i=1}^n||\mathbf{x}_i||\quad\text{where $n$ is a finite integer}
	\end{align}
\end{proposition}

\subsection{Single Line}\index{Propositions!Single Line}

\begin{proposition} % Specify a name/title in square brackets, or leave them out for no title
	Let $f,g\in L^2(G)$; if $\forall \varphi\in\mathcal{D}(G)$, $(f,\varphi)_0=(g,\varphi)_0$ then $f = g$. 
\end{proposition}

%------------------------------------------------

\section{Examples}\index{Examples}

\subsection{Equation Example}\index{Examples!Equation}

\begin{example} % Specify a name/title in square brackets, or leave them out for no title
	Let $G=\{x\in\mathbb{R}^2:|x|<3\}$ and denoted by: $x^0=(1,1)$; consider the function:
	\begin{equation}
	f(x)=\left\{\begin{aligned} & \mathrm{e}^{|x|} & & \text{si $|x-x^0|\leq 1/2$}\\
	& 0 & & \text{si $|x-x^0|> 1/2$}\end{aligned}\right.
	\end{equation}
	The function $f$ has bounded support, we can take $A=\{x\in\mathbb{R}^2:|x-x^0|\leq 1/2+\epsilon\}$ for all $\epsilon\in\mathopen{]}0\,;5/2-\sqrt{2}\mathclose{[}$.
\end{example}

\subsection{Text Example}\index{Examples!Text}

\begin{example}[Example name] % Specify a name/title in square brackets, or leave them out for no title
	Aliquam arcu turpis, ultrices sed luctus ac, vehicula id metus. Morbi eu feugiat velit, et tempus augue. Proin ac mattis tortor. Donec tincidunt, ante rhoncus luctus semper, arcu lorem lobortis justo, nec convallis ante quam quis lectus. Aenean tincidunt sodales massa, et hendrerit tellus mattis ac. Sed non pretium nibh. Donec cursus maximus luctus. Vivamus lobortis eros et massa porta porttitor.
\end{example}

%------------------------------------------------

\section{Exercises}\index{Exercises}

\begin{exercise} % Specify a name/title in square brackets, or leave them out for no title
	This is a good place to ask a question to test learning progress or further cement ideas into students' minds.
\end{exercise}

%------------------------------------------------

\section{Problems}\index{Problems}

\begin{problem} % Specify a name/title in square brackets, or leave them out for no title
	What is the average airspeed velocity of an unladen swallow?
\end{problem}

%------------------------------------------------

\section{Vocabulary}\index{Vocabulary}

Define a word to improve a students' vocabulary.

\begin{vocabulary}[Word] % Specify a name/title in square brackets, or leave them out for no title
	Definition of word.
\end{vocabulary}
