\chapterimage{orange3.jpg} % Chapter heading image
\chapterspaceabove{6.25cm} % Whitespace from the top of the page to the chapter title on chapter pages
\chapterspacebelow{7.5cm} % Amount of vertical whitespace from the top margin to the start of the text on chapter pages

%------------------------------------------------

\chapter{Presenting Information and Results with a Long Chapter Title}

\section{Table}\index{Table}

Lorem ipsum dolor sit amet, consectetur adipiscing elit. Praesent porttitor arcu luctus, imperdiet urna iaculis, mattis eros. Pellentesque iaculis odio vel nisl ullamcorper, nec faucibus ipsum molestie. Sed dictum nisl non aliquet porttitor. Etiam vulputate arcu dignissim, finibus sem et, viverra nisl. Aenean luctus congue massa, ut laoreet metus ornare in. Nunc fermentum nisi imperdiet lectus tincidunt vestibulum at ac elit. Nulla mattis nisl eu malesuada suscipit.

\begin{table}[H] % Use [H] to suppress floating and place the figure/table exactly where it is specified in the text
	\centering % Horizontally center the table on the page
	\begin{tabular}{L{0.15\textwidth} R{0.15\textwidth} R{0.15\textwidth}} % Specify column alignment with L{width}, C{width} and R{width} for fixed-width columns, or the default latex l, c and r for flexible-width columns
		\toprule
		\textbf{Treatments} & \textbf{Response 1} & \textbf{Response 2}\\
		\midrule
		Treatment 1 & 0.0003262 & 0.562 \\
		Treatment 2 & 0.0015681 & 0.910 \\
		Treatment 3 & 0.0009271 & 0.296 \\
		\bottomrule
	\end{tabular}
	\caption{Table caption.}
	\label{tab:example} % Unique label used for referencing the table in-text
\end{table}

Referencing \autoref{tab:example} in-text using its label.

\begin{table}[t] % Floating table, [t] tells LaTeX to place it at the top of the next available page
	\centering % Horizontally center the table on the page
	\begin{tabular}{L{0.15\textwidth} R{0.15\textwidth} R{0.15\textwidth}} % Specify column alignment with L{width}, C{width} and R{width} for fixed-width columns, or the default latex l, c and r for flexible-width columns
		\toprule
		\textbf{Treatments} & \textbf{Response 1} & \textbf{Response 2}\\
		\midrule
		Treatment 1 & 0.0003262 & 0.562 \\
		Treatment 2 & 0.0015681 & 0.910 \\
		Treatment 3 & 0.0009271 & 0.296 \\
		\bottomrule
	\end{tabular}
	\caption{Floating table.}
	\label{tab:floating} % Unique label used for referencing the table in-text
\end{table}

%------------------------------------------------

\section{Figure}\index{Figure}

Lorem ipsum dolor sit amet, consectetur adipiscing elit. Praesent porttitor arcu luctus, imperdiet urna iaculis, mattis eros. Pellentesque iaculis odio vel nisl ullamcorper, nec faucibus ipsum molestie. Sed dictum nisl non aliquet porttitor. Etiam vulputate arcu dignissim, finibus sem et, viverra nisl. Aenean luctus congue massa, ut laoreet metus ornare in. Nunc fermentum nisi imperdiet lectus tincidunt vestibulum at ac elit. Nulla mattis nisl eu malesuada suscipit.

\begin{figure}[H] % Use [H] to suppress floating and place the figure/table exactly where it is specified in the text
	\centering % Horizontally center the figure on the page
	\includegraphics[width=0.5\textwidth]{creodocs_logo.pdf} % Include the figure image
	\caption{Figure caption.}
	\label{fig:placeholder} % Unique label used for referencing the figure in-text
\end{figure}

Referencing \autoref{fig:placeholder} in-text using its label.

\begin{figure}[b] % Floating figure, [b] tells LaTeX to place it at the bottom of the next available page
	\centering % Horizontally center the figure on the page
	\includegraphics[width=\textwidth]{creodocs_logo.pdf} % Include the figure image
	\caption{Floating figure.}
	\label{fig:floating} % Unique label used for referencing the figure in-text
\end{figure}